%----------------------------------------------------------------------------------------
%	CONFIGURATIONS
%----------------------------------------------------------------------------------------

\documentclass[12pt,a4paper,oneside]{article}

\usepackage[utf8]{inputenc}
\usepackage{graphicx}
\usepackage{epstopdf}
\usepackage{natbib}
\usepackage{amsmath}
\usepackage{lipsum}
\usepackage{caption}
\usepackage{subcaption}
\usepackage[a4paper,left=2cm,right=2cm,top=2.5cm,bottom=2.5cm]{geometry}

%----------------------------------------------------------------------------------------
%	INFORMATION
%----------------------------------------------------------------------------------------

\title{Estudo de paralelismo no problema \textit{All-Pairs Shortest
    Paths}}

\author{Filipe Figueiredo\footnote{Filipe Figueiredo - 201203559},
  Pedro Paredes\footnote{Pedro Paredes - 201205725}, DCC - FCUP}

\date{Novembro 2015}

\renewcommand{\tablename}{Tabela}
\renewcommand{\figurename}{Figura}
\renewcommand{\refname}{Referências}

\begin{document}

\maketitle

%----------------------------------------------------------------------------------------
%	SECTION 1
%----------------------------------------------------------------------------------------

\section{Introdução}
\label{sec:intro}
O problema do caminho mais curto (ou \textit{Shortest Path} em inglês)
é um problema clássico com grande aplicabilidade. O caminho mais curto
entre dois nós $u$ e $v$ consiste em determinar o caminho com vértice
inicial $u$ e vértice final $v$, tal que o custo do caminho é o menor
possível. O custo do caminho num grafo não pesado é determinado pelo
número de arestas que o caminho contém e num grafo pesado é
determinado pela soma dos pesos das arestas do caminho.

Historicamente o problema foi estudado usando duas estratégias
diferentes, para as quais existem vários algoritmos. A primeira,
chamada de \textit{Single-source shortest path} (ou caminho mais curto
a partir de um só vértice) tem como objetivo encontrar todos os
caminhos mais curtos que começam num determinado nó $u$. Já o
\textit{All-pairs shortest paths} (ou todos os pares de caminhos mais
curtos) tem como objetivo encontrar os caminhos mais curtos entre
todos os possíveis pares de nós.

Neste trabalho focamo-nos em usar a segunda estratégia em grafos
pesados. O objetivo será estudar o paralelismo existente em diferentes
algoritmos e obter uma implementação eficiente de modo a verificar a
sua escalabilidade.

O resto do relatório está organizado da seguinte forma. A
Secção~\ref{sec:ai} descreve em detalhe os algoritmos a estudar e as
suas versões paralelas. A Secção~\ref{sec:res} apresenta o estudo
empírico dos algoritmos através de testes com diferentes
redes. Finalmente, na Secção~\ref{sec:con} é feita uma breve conclusão
e apresentadas algumas notas finais.


%----------------------------------------------------------------------------------------
%	SECTION 2
%----------------------------------------------------------------------------------------

\section{Algoritmos e implementação}
\label{sec:ai}


%----------------------------------------------------------------------------------------
%	SECTION 3
%----------------------------------------------------------------------------------------

\section{Resultados e discussão}
\label{sec:res}

%----------------------------------------------------------------------------------------
%	SECTION 4
%----------------------------------------------------------------------------------------

\section{Conclusão e notas finais}
\label{sec:con}

\cite{shimbel1953structural}

\bibliographystyle{plain}
\bibliography{report}

\end{document}
